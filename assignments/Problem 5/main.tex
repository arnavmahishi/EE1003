\let\negmedspace\undefined
\let\negthickspace\undefined
\documentclass[journal]{IEEEtran}
\usepackage[a5paper, margin=10mm, onecolumn]{geometry}
\usepackage{lmodern} % Ensure lmodern is loaded for pdflatex
\usepackage{tfrupee} % Include tfrupee package

\setlength{\headheight}{1cm} % Set the height of the header box
\setlength{\headsep}{0mm}     % Set the distance between the header box and the top of the text

\usepackage{gvv-book}
\usepackage{gvv}
\usepackage{cite}
\usepackage{amsmath,amssymb,amsfonts,amsthm}
\usepackage{algorithmic}
\usepackage{graphicx}
\usepackage{textcomp}
\usepackage{xcolor}
\usepackage{txfonts}
\usepackage{listings}
\usepackage{enumitem}
\usepackage{mathtools}
\usepackage{gensymb}
\usepackage{comment}
\usepackage[breaklinks=true]{hyperref}
\usepackage{tkz-euclide} 
\usepackage{listings}                                      
\def\inputGnumericTable{}                                 
\usepackage[latin1]{inputenc}                                
\usepackage{color}                                            
\usepackage{array}                                            
\usepackage{longtable}
\usepackage{multicol}
\usepackage{calc}                                             
\usepackage{multirow}                                         
\usepackage{hhline}                                           
\usepackage{ifthen}                                           
\usepackage{lscape}
\begin{document}

\bibliographystyle{IEEEtran}
\vspace{3cm}

\title{10.4.1.2.2}
\author{EE24BTECH11006 - Arnav Mahishi}
% \maketitle
% \newpage
% \bigskip
{\let\newpage\relax\maketitle}

\renewcommand{\thefigure}{\theenumi}
\renewcommand{\thetable}{\theenumi}
\setlength{\intextsep}{10pt} % Space between text and floats


\numberwithin{equation}{enumi}
\numberwithin{figure}{enumi}
\renewcommand{\thetable}{\theenumi}


\textbf{Question}:\newline
The product of two consecutive integers is $306$. We need to find the integers.
\newline
\begin{table}[h!]    
  \centering
  \begin{tabular}[10pt]{ |c| c| c|}
    \hline
    \textbf{input}&\textbf{Description}&\textbf{value}\\
    \hline 
    $a$&Length of semi major axis of ellipse&$3$\\
    \hline
    $b$&Length of semi minor axis of ellipse&$2$\\
    \hline
    $v$&Quadratic form of matrix&$\myvec{b^2&0\\0&a^2}$\\
    \hline 
    $u$&Linear coefficient vector&$0$\\
    \hline 
    $f$&Constant Term&$-(a^2b^2)$\\
    \hline
    $h$&One of the points the line passes through&$\myvec{a\\0}$\\
    \hline
    $m$&Slope of line&$\myvec{\frac{1}{b}\\\frac{-1}{a}}$\\
    \hline
    $n$& number of subintervals we are taking & $1000$\\
    \hline
    $x_0$&$x$ coordinate of first intersection point& $3$\\
    \hline
    $x_n$& $y$ coordinate of second intersection point& $2$\\
    \hline
    \end{tabular}

  \caption{Variables Used}
  \label{tab1.1.2.2}
\end{table}
\newline
\textbf{Theoretical Solution:}\\
Lets start by assuming the bigger integer as $x$ and the smaller integer as $\frac{306}{x}$
\begin{align}
    \implies x-\frac{306}{x}=1\\
    \implies x^2-306=x\\
    \implies x^2-x-306=0
\end{align}
Using the quadratic formula:
\begin{align}
    x=\frac{1\pm\sqrt{1^2-\brak{4\cdot -306}}}{2}\\
    x_1=\frac{1+\sqrt{1225}}{2}=18\\
    x_2=\frac{1-\sqrt{1225}}{2}=-17
\end{align}
If $x=18$ the other integer will be $17$ if $x=-17$ the other integer will be $-18$
\begin{align}
    p\brak{2}=41-72\brak{-2}-18\brak{-2}^2=113
\end{align}
$\therefore$ The integers can be $18,17$ or $-18,-17$\\
\textbf{Computational Solution:}\\
Below are three methods to find the solutions of this quadratic equation,\\
Matrix-Based Method:\\
For a polynomial equation of form $x_n+b_{n-1}x^{n-1}+\dots+b_2x^2+b_1x+b_0 = 0$ we construct a matrix called companion matrix of form
\begin{align}
	\Lambda = \myvec{0&1&0&\dots&0\\ 0&0&1&\dots&0\\ \vdots &\vdots &\vdots &\ddots&\vdots\\0&0&0&\vdots&1\\-b_0&-b_1&-b_2&\dots&-b_{n-1}}
\end{align}
The eigenvalues of this matrix are the roots of the given polynomial equation.\\
The solution given by the code is
\begin{align}
	x_1 = 18.000000\\
	x_2 = -17.00000\
\end{align}
Fixed Point Iterations:
Take an initial guess $x_0$. The update difference equation will use the following function:
\begin{align}
    x = g\brak{x}
\end{align}
For our problem,
\begin{align}
    g\brak{x} = \sqrt{x^2-306}
\end{align}
To get both roots we do two iterations, now the update equations will be
\begin{align}
    x_{\alpha,n+1}=g\brak{x_{\alpha,n}}\\
    x_{\beta,n+1}=g\brak{x_{\beta,n}}\\
\end{align}
We take two iniital guesses $x_{\alpha,0}$ and $x_{\beta,0}$ close to each root. Then we continue calculating the each $x_{\alpha,n}$ and $x_{\beta,n}$ until
\begin{align}
    \abs{x_{n+1}-x_n}<\epsilon
\end{align}
Where $\epsilon$ is the tolerance which we have taken as $1\text{e-}6$. In each of the $\alpha$ series and $\beta$ series we get a root.\\
\newline
This is proved by a theorem as follows:\\
Let $x = s$ be a solution of $x = g\brak{x}$ and suppose that $g$ has a continuous
derivative in some interval $J$ containing $s$. Then if $\abs{g^{\prime}} \le K < 1$ in $J$,
the iteration process defined  above converges for any $x_0$ in $J$. The limit of the sequence
$\sbrak{x_n}$ is s.\\
\newline
This can also be solved by the Newton-Raphson Method,\\
Start with an initial guess $x_0$, and then run the following logical loop,
\begin{align}
    x_{n+1} = x_n - \frac{f\brak{x_n}}{f^{\prime}\brak{x_n}} 
\end{align}
where ,
\begin{align}
    f\brak{x} = x^2 - x - 306\\
    f^{\prime}\brak{x} = 2x-1
\end{align}
\newline
The problem with this method is if the roots are complex but the coeffcients are real, $x_n$ either converges to an extrema or grows continuously without any bound.
To get the complex solutions, however , we can just take the initial guess point to be a 
random complex number.\\
The output of a program written to find roots is shown below:
\begin{verbatim}
Fixed-Point Iteration for Positive Root:
Iteration 1: x = 17.577532, f(x) = -14.607907
Iteration 2: x = 17.988261, f(x) = -0.410729
Iteration 3: x = 17.999674, f(x) = -0.011413
Iteration 4: x = 17.999996, f(x) = -0.000237
Iteration 5: x = 18.000000, f(x) = -0.000009
Converged to solution: x = 18.000000

Fixed-Point Iteration for Negative Root:
Iteration 1: x = -9.767519, f(x) = -200.828057
Iteration 2: x = -17.211406, f(x) = 7.443887
Iteration 3: x = -16.993781, f(x) = 0.217625
Iteration 4: x = -17.001183, f(x) = -0.006402
Iteration 5: x = -16.999995, f(x) = 0.000018
Iteration 6: x = -17.000000, f(x) = -0.000000
Converged to solution: x = -17.000000

Newton-Raphson Iteration for Positive Root:
Iteration 1: x = 19.448724, f(x) = 52.804159
Iteration 2: x = 18.055381, f(x) = 1.941406
Iteration 3: x = 18.000887, f(x) = 0.003057
Iteration 4: x = 18.000001, f(x) = 0.000007
Iteration 5: x = 18.000000, f(x) = 0.000000
Converged to solution: x = 18.000000

Newton-Raphson Iteration for Negative Root:
Iteration 1: x = -19.809749, f(x) = 106.235914
Iteration 2: x = -17.194357, f(x) = 6.840276
Iteration 3: x = -17.007687, f(x) = 0.073361
Iteration 4: x = -17.000017, f(x) = 0.000307
Iteration 5: x = -17.000000, f(x) = 0.000000
Converged to solution: x = -17.000000

Eigenvalues (Roots of the equation):
Eigenvalue 1: -17.000000 +0.0000000i
Eigenvalue 2: 18.000000 + 0.0000000i
\end{verbatim}

\end{document}

